\documentclass[12pt]{article}
\usepackage{enumerate}
\usepackage{amsmath}
\usepackage{amsthm}
\usepackage{amssymb}
\usepackage{changepage}
\usepackage{graphicx}

\newcommand{\ytilde}{\tilde{y}}
\newcommand{\X}{\mathbf{X}}
\newcommand{\btilde}{\tilde{\beta}}
\newcommand{\etilde}{\tilde{\epsilon}}
\newcommand{\lm}{\ytilde = \X\btilde + \etilde}
\newcommand{\exE}{E[\etilde]}
\newcommand{\etet}{E[\etilde\etilde^{\T}]}
\newcommand{\btildematrix}{(\X^{\T}\X)^{-1}\X^{\T} \ytilde}
\newcommand{\xtx}{\X^{\T}\X}
\newcommand{\xtxinv}{(\X^{\T}\X)^{-1}}
\newcommand{\I}{\mathbf{I}}
\newcommand{\lamvector}{\lambda_{1}, \lambda_{2}, \ldots, \lambda_{p}}
\newcommand{\T}{\mathbf{T}}
\newcommand{\HM}{\mathbf{H}}


\title{Assignment 3}
\author{Rin Meng \\ Student ID: 51940633}
\begin{document}
\maketitle
\textbf{Note:} From this point onwards I will use tilde on top of a variable to denote that it is a vector.
    \begin{enumerate}[1.]
        \item Given the linear regression model in the form $\lm$ where $\exE = 0$ and $\etet = \sigma^2I$, and $\etilde$ is 
        normally distributed, which implies that the least-squares estimator for $\btilde$
        \[\btilde = \btildematrix\]
        Where we require the matrix ($\xtx$) to be invertible.
            \begin{enumerate}[a)]
                \item Show that $\xtx + \lambda \I$ is invertible, that is $\text{det}(\xtx + \lambda \I) \neq 0$ 
                when $\lambda \neq 0$, and $\I$ is the identity matrix.
                \begin{proof}
                    We know that $\xtx$ is non-invertible, that is $\text{det}(\xtx) = 0$, and $a$ and $d$ are non-negative.
                    \begin{align*}
                        \xtx + \lambda \I &=  \begin{bmatrix} a & b \\ c & d \end{bmatrix} + 
                        \lambda \begin{bmatrix} 1 & 0 \\ 0 & 1 \end{bmatrix} \\
                        &= \begin{bmatrix} a + \lambda & b \\ c & d + \lambda \end{bmatrix}
                    \end{align*}
                    Using the formula of determinant of $2 \times 2$ matrix, we have
                    \begin{align*}
                        \text{det}(\xtx + \lambda \I) &= \text{det}\left(
                            \begin{bmatrix} a + \lambda & b \\ c & d + \lambda \end{bmatrix}\right) \\
                        &= (a + \lambda)(d + \lambda) - bc \\
                        &= ad + a\lambda + d\lambda + \lambda^2 - bc \\
                        &= ad + \lambda(a + d) + \lambda^2 - bc
                    \end{align*}
                    Given that $\text{det}(\xtx) = 0 \Rightarrow ad - bc = 0 \Leftrightarrow ad = bc$.
                    \begin{align*}
                        \text{det}(\xtx + \lambda \I) &= bc + \lambda(a + d) + \lambda^2 - bc \\
                        &= \lambda(a + d) + \lambda^2
                    \end{align*}
                    \begin{align*} 
                        \because \lambda \neq 0 &\Rightarrow \lambda^2 > 0 \\
                        \because a, d \geq 0 &\Rightarrow a + d > 0 \\
                        &\Rightarrow \lambda(a + d) + \lambda^2 > 0
                    \end{align*}
                
                    $\therefore$ We have shown that $\text{det}(\xtx + \lambda \I) \neq 0$ when $\lambda \neq 0$.
                \end{proof}

                \item Given the expression 
                \[(\ytilde - \X \btilde)^\T(\ytilde - \X \btilde) + \lambda \btilde^\T \btilde\]
                where $\lambda > 0$

                Let us expand the expression algebraically, using what we learned about minimizing 
                $(\ytilde - \X \btilde)^\T(\ytilde - \X \btilde)$ to obtain the estimator $\btilde$ in terms of $\X$, $\ytilde$ and $\lambda$.
                \begin{align*}
                    (\ytilde - \X \btilde)^\T(\ytilde - \X \btilde) &= (\ytilde^\T - \btilde^\T \X^\T)(\ytilde - \X \btilde) \\
                    &= \ytilde^\T \ytilde - \ytilde^\T \X \btilde - \btilde^\T \X^\T \ytilde + \btilde^\T \X^\T \X \btilde \\
                    &= \ytilde^\T \ytilde - 2\btilde^\T \X^\T \ytilde + \btilde^\T \X^\T \X \btilde\\ 
                    (\ytilde - \X \btilde)^\T(\ytilde - \X \btilde) + \lambda \btilde^\T \btilde 
                    &= \ytilde^\T \ytilde - 2\btilde^\T \X^\T \ytilde + \btilde^\T \X^\T \X \btilde + \lambda \btilde^\T \btilde \\
                    &= \ytilde^\T \ytilde - 2\btilde^\T \X^\T \ytilde + \btilde^\T (\X^\T \X + \lambda \I) \btilde
                \end{align*}
                Now let us minimize this expression my deriving it with respect to $\btilde$ and setting it to zero.
                \begin{align*}
                    \frac{\partial}{\partial \btilde} ( \ytilde^\T \ytilde  + 2\btilde^\T \X^T \ytilde + \btilde^\T (\xtx + \lambda \I)\btilde) &= 0 \\
                    - 2 \X^\T \ytilde + (\X^\T \X + \lambda \I) \btilde &= 0 \\
                    \X^\T \ytilde &= (\X^\T \X + \lambda \I) \btilde \\
                    \btilde &= (\X^\T \X + \lambda \I)^{-1} \X^\T \ytilde
                \end{align*}

                $\therefore$ The estimator $\btilde$ in terms of $\X$, $\ytilde$ and $\lambda$ is 
                $(\X^\T \X + \lambda \I)^{-1} \X^\T \ytilde$.

                \item Show that the estimator $\btilde$ is biased.
                \begin{align*}
                    (\X^\T \X + \lambda \I)^{-1} \X^\T \ytilde &= (\X^\T \X + \lambda \I)^{-1} \X^\T (\X \beta + \epsilon) \\
                    &= (\X^\T \X + \lambda \I)^{-1} \X^\T \X \beta + (\X^\T \X + \lambda \I)^{-1} \X^\T \epsilon \\
                    \text{Let } \mathbf{A} & = (\X^\T \X + \lambda \I)^{-1} \X^\T \X \\
                    \text{E}[\btilde] &= \text{E}[A \beta] + (\X^\T \X + \lambda \I)^{-1} \X^\T \text{E}[\epsilon] \\
                    \text{E}[\btilde] &= \text{E}[A \beta]
                \end{align*}
                For $\btilde$ to be unbiased, we need $\text{E}[\btilde] = \beta \Leftrightarrow \text{E}[\mathbf{A} \beta] = \beta \leftrightarrow \mathbf{A} = \I$,
                but we know that this is now true because $\mathbf{A} = (\X^\T \X + \lambda \I)^{-1} \X^\T \X \neq \I$.

                $\therefore$ The estimator $\btilde$ is biased.
            \end{enumerate}

            \item Given that $H_0: \beta_2 = \beta_6 = 0$, and $H_0: \beta_2 = \beta_6$, $\beta_3 = \beta_4$.
                \begin{enumerate}[a)]
                    \item A matrix T can rpresent the null hypothesis $H_0: \beta_2 = \beta_6 = 0$ as
                    \begin{align*}
                        T &= \begin{bmatrix} 0 & 1 & 0 & 0 & 0 & 0 \\ 0 & 0 & 0 & 0 & 0 & 1 \end{bmatrix}
                    \end{align*}
                    \item A matrix T can represent the null hypothesis $H_0: \beta_2 = \beta_6$, $\beta_3 = \beta_4$ as
                    \begin{align*}
                        T &= \begin{bmatrix} 0 & 1 & 0 & 0 & 0 & -1 \\ 0 & 0 & 1 & -1 & 0 & 0 \end{bmatrix}
                    \end{align*}
                    Using the fact that $\beta_2 = \beta_6 \Leftrightarrow \beta_2 - \beta_6 = 0$ and $\beta_3 = \beta_4 \Leftrightarrow \beta_3 - \beta_4 = 0$.
                    
                    \item Under $H_0: \T\btilde = 0$, show that $\text{Var}(\T \btilde) = \sigma^2 \T (\X^\T \X)^{-1} \T^\T$.
                        \begin{proof}
                            \begin{align*}
                                \text{Var}(\T \btilde) &= \text{Var}(\T (\X^\T \X)^{-1} \X^\T \etilde) \\
                                &= \T (\X^\T \X)^{-1} \X^\T  \cdot \text{Var}(\etilde) \cdot \X (\X^\T \X)^{-1} \T^\T \\
                                &= \T (\X^\T \X)^{-1} \X^\T \cdot \sigma^2 \I \cdot \X (\X^\T \X)^{-1} \T^\T \\
                                &= \sigma^2 \T (\X^\T \X)^{-1} \X^\T \X (\X^\T \X)^{-1} \T^\T \\
                                &= \sigma^2 \T (\X^\T \X)^{-1} \T^\T
                            \end{align*}
                            $\therefore$ We have shown that $\text{Var}(\T \btilde) = \sigma^2 \T (\X^\T \X)^{-1} \T^\T$.
                        \end{proof}
                    \item Under $H_0: \T\btilde = 0$, show that 
                        $\T \btilde = \T (\X^\T \X)^{-1} \X^\T \etilde$.
                        \begin{proof}
                            \begin{align*}
                                \T \btilde &= \T (\X^\T \X)^{-1} \X^\T \ytilde \\
                                &= \T (\X^\T \X)^{-1} \X^\T (\X \beta + \epsilon) \\
                                &= \T (\X^\T \X)^{-1} \X^\T \X \beta + \T (\X^\T \X)^{-1} \X^\T \epsilon \\
                                &= \T \beta + \T (\X^\T \X)^{-1} \X^\T \epsilon \\
                                &= \T (\X^\T \X)^{-1} \X^\T \epsilon
                            \end{align*}
                            $\therefore$ We have shown that $\T \btilde = \T (\X^\T \X)^{-1} \X^\T \etilde$.
                        \end{proof}

                    \item Under $H_0: \T\btilde = 0$, show that $\hat{\btilde}^\T \T^\T \Sigma^{-1} \T\hat{\btilde} \sim \chi_{(r)}^2$, where 
                    $\Sigma^{-1} = \text{Var}(\T\hat{\btilde})^{-1}$.

                    \begin{proof}
                        Recall that the $\hat{\btilde} = (\X^\T \X)^{-1} \X^\T \ytilde$, where $\ytilde = \X \beta + \etilde$.
                        The variance of $\T \hat{\btilde}$ is $\text{Var}(\T \hat{\btilde}) = \sigma^2 \T (\X^\T \X)^{-1} \T^\T$.
                        Thus, 
                        \begin{align*}
                            \Sigma &= \sigma^2 \T (\X^\T \X)^{-1} \T^\T \\
                            \Sigma^{-1} &= \frac{1}{\sigma^2} (\T (\X^\T \X)^{-1} \T^\T)^{-1}
                        \end{align*}
                        Now we have
                        \begin{align*}
                            \hat{\btilde}^\T \T^\T \Sigma^{-1} \T \hat{\btilde} 
                            &= \hat{\btilde}^\T \T^\T \left(\frac{1}{\sigma^2} (\T (\X^\T \X)^{-1} \T^\T)^{-1}\right) \T \hat{\btilde} \\
                            &= \frac{1}{\sigma^2} \hat{\btilde}^\T \T^\T (\T (\X^\T \X)^{-1} \T^\T)^{-1} \T \hat{\btilde}
                        \end{align*}
                        Let us go back to our null hypothesis $H_0: \T \btilde = 0$. We know that the $\hat{\btilde}$ 
                        is normally distributed with mean 0 and covariance matrix $\sigma^2 (\X^\T \X)^{-1}$, therefore so is
                        $\T \hat{\btilde}$. Then we have
                        \[\T\hat{\btilde} \sim N(\T\beta, \sigma^2 \T (\X^\T \X)^{-1} \T^\T)\]
                        Under $H_0$ this becomes
                        \[\T\hat{\btilde} \sim N(0, \sigma^2 \T (\X^\T \X)^{-1} \T^\T)\]
                        Now, let S be the statistic of our test, then
                        \[S = \frac{1}{\sigma^2} \hat{\btilde}^\T \T^\T (\T (\X^\T \X)^{-1} \T^\T)^{-1} \T \hat{\btilde} \]
                        then $S$ is definetly in a form of $\chi^2$ distribution, we can show by letting
                        \[\textbf{z} = \T \hat{\btilde}\]
                        \[\textbf{A} = \frac{1}{\sigma^2} (\T (\X^\T \X)^{-1} \T^\T)^{-1} \]
                        and that implies, 
                        \[S = \textbf{z}^\T A \textbf{z}\]
                        where $\text{z} \sim N(0, \sigma^2 \T (\X^\T \X)^{-1} \T^\T)$, as shown above. The rank of 
                        $\T$ is $r = \textbf{rank}(\T)$, which represents the number of linearly independent rows in $\T$, where it
                        ultimately implies that the degrees of freedom of $\chi^2$ distribution is $r$.

                        $\therefore$ We have shown that 
                        $\hat{\btilde}^\T \T^\T \Sigma^{-1} \T \hat{\btilde} \sim \chi_{(\textbf{rank}(\textbf{T}))}^2 = \chi_{(r)}^2$.
                    \end{proof}
                    \item Show that 
                    \[(\I - \HM)[\X \xtxinv \T^\T \textbf{C}^{-1} \T \xtxinv \X] = 0\]

                    \begin{proof}
                        We know that $\HM = \X \xtxinv \X^\T$, and $\textbf{C} = \T (\X^\T \X)^{-1} \T^\T$. The term 
                        $(\I - \HM)$ is the projection matrix that projects onto 
                        the orthogonal complement of the column space of $\X$. Thus we get:
                        \[ (\I - \HM) = \I - \X \xtxinv \X^\T \]
                        \[ (\I - \HM)[\X \xtxinv \T^\T \textbf{C}^{-1} \T \xtxinv \X] \]
                        \[= \X \xtxinv \T^\T \textbf{C}^{-1} \T \xtxinv \X \]
                        \[- \HM (\X \xtxinv \T^\T \textbf{C}^{-1} \T \xtxinv \X)\]
                        and since $\HM\X = \X$, we have
                        \[= \X \xtxinv \T^\T \textbf{C}^{-1} \T \xtxinv \X \]
                        \[- \X \xtxinv \T^\T \textbf{C}^{-1} \T \xtxinv \X\]
                        \[= 0\]
                        
                        $\therefore$ We have shown that $(\I - \HM)[\X \xtxinv \T^\T \textbf{C}^{-1} \T \xtxinv \X] = 0$.
                    \end{proof}

                    \item Under $H_0: \T\btilde = 0$, show that 
                    \[F_0 = \frac{(\hat{\btilde}^\T \T^\T \textbf{C}^{-1} \T \hat{\btilde}) / r}{MSE} \sim F_{(r, n - p)}\] 
                    where MSE is computed for the full model (with p parameters)

                    From earlier proof, we know that $\hat{\btilde}^\T \T^\T \textbf{C}^{-1} \T \hat{\btilde} \sim \chi_{(r)}^2$.
                    We also know that $MSE = \frac{SSE}{n - p} \Leftrightarrow MSE \sim \chi^2_{(n - p)}$.
                    Then we have something like this
                    \[F_0 = \frac{(\hat{\btilde}^\T \T^\T \textbf{C}^{-1} \T \hat{\btilde}) / r}{SSE / (n - p)} \] 
                    where we notice that this is a form of $F$ distribution, 
                    where both numerator and denominator are some sort of $\chi^2$ distribution now visually we can see that
                    \[= \frac{\sigma^2 \chi^2_{r}}{\sigma^2 \chi^2_{n-p}} = \frac{\chi^2_{r}}{\chi^2_{n-p}} 
                    \sim F_{(r, n - p)}\]

                    \item Find a matrix $\T$ that represents some hypothesis: 
                    \[H_\gamma : \beta_0 = \beta_1 = \beta_2 = \cdots = \beta_k\]
                    
                    \begin{proof}
                        Let us consider the hypothesis 
                        \[H_\gamma : \beta_0 = \beta_1 = \beta_2 = \cdots = \beta_k = \beta\]
                        (here we let $\beta$ can represent a common value for all the parameters).
                        We can represent this hypothesis as 
                        \[\beta_0 - \beta = 0, \beta_1 - \beta = 0, \beta_2 - \beta = 0, \ldots, \beta_k - \beta = 0\]
                        and our goal is to present this a matrix $\T$ such that 
                        \[\T \beta = 0\]
                        where $\beta = [\beta_0, \beta_1, \ldots, \beta_k]^\T$.

                        Now lets start constructing the matrix $\T$, with these constraints:
                        \begin{enumerate}
                            \item We need the matrix $\T$ to be the size of $k_{\text{rows}} \times (k + 1)_{\text{column}}$, where $k$ is the number of parameters and the 
                            $ + 1$ is reserved for the intercept term.
                            \item The first column of $\T$ represents constraints for $\beta_0$, 
                            the second column represents constraints for $\beta_1$, and so on.
                            \item Each row of $\T$ represents the constraints of a parameter in some form $\beta_i - \beta = 0$. 
                            We can write this as $\T \beta = \mathbf{c}$.
                        \end{enumerate}
                        \[\T = 
                            \begin{bmatrix} 
                                   1 & -1 & 0 & 0 & \cdots & 0 
                                \\ 0 & 1 & -1 & 0 & \cdots & 0 
                                \\ 0 & 0 & 1 & -1 & \cdots & 0 
                                \\ \vdots & \vdots & \vdots & \vdots & \ddots & \vdots 
                                \\ 0 & 0 & 0 & 0 & \cdots & 1 
                            \end{bmatrix}
                        \]
                        then the $\T \beta$ matrix will look like this
                        \[\T \beta = 
                            \begin{bmatrix} 
                                   1 & -1 & 0 & 0 & \cdots & 0 
                                \\ 0 & 1 & -1 & 0 & \cdots & 0 
                                \\ 0 & 0 & 1 & -1 & \cdots & 0 
                                \\ \vdots & \vdots & \vdots & \vdots & \ddots & \vdots 
                                \\ 0 & 0 & 0 & 0 & \cdots & 1 
                            \end{bmatrix}
                            \begin{bmatrix} 
                                \beta_0 \\ \beta_1 \\ \beta_2 \\ \vdots \\ \beta_k
                            \end{bmatrix}
                            = 
                            \begin{bmatrix} 
                                \beta_0 - \beta_1 \\ \beta_1 - \beta_2 \\ \beta_2 - \beta_3 \\ \vdots \\ \beta_{k-1} - \beta_k
                            \end{bmatrix}
                            =
                            \begin{bmatrix}
                                0 \\ 0 \\ 0 \\ \vdots \\ 0
                            \end{bmatrix}
                        \]
                        which implies that $\mathbf{c} = 0 \Rightarrow \T \beta = 0$.
                        $\therefore$ We have shown that the matrix $\T$ that represents the hypothesis $H_\gamma : \beta_0 = \beta_1 = \beta_2 = \cdots = \beta_k$ is
                    \end{proof}
                \end{enumerate}
                \item Problem 3.25 on page 130, using ``lm'' function to answer the following questions. We are given that the linear regression
                model is 
                \[ y = \beta_0 + \beta_1 x_1 + \beta_2 x_2 + \beta_3 x_3 + \beta_4 x_4 +  \epsilon\]
                \begin{enumerate}[a.]
                    \item $H_0: \beta_1 = \beta_2 = \beta_3 = \beta_4 = \beta 
                    \Leftrightarrow \beta_1 - \beta = 0, \beta_2 - \beta = 0, \beta_3 - \beta = 0, \beta_4 - \beta = 0$.
                    \begin{verbatim}
# Assuming that the data we are using is table.b1
# are the columns/predictors
# Load required library to use linearHypothesis()
library(car)
model = lm(y ~ x1 + x2 + x3 + x4, data = table.b1)

# testing the hypothesis
linear_hypothesis_test(model, "x1 = x2 = x3 = x4")
                    \end{verbatim}

                    \begin{verbatim}
# function to perform hypothesis test using 
# linearHypothesis()
linear_hypothesis_test <- function(model) {

# use matrix to specify the hypothesis and constraintss
# from the last proof we did in the previous question
hypothesis_matrix <- matrix(c(1, -1, 0, 0, 0, 
                              0, 1, -1, 0, 0, 
                              0, 0, 1, -1, 0,
                              0, 0, 0, 1, -1),
                nrow = 4, ncol = 5, byrow = TRUE)

# specify the hypothesis (all coefficients equal to 0)
hypothesis_values <- c(0, 0, 0, 0)

# perform the linear hypothesis test
linear_hypothesis_result <- linearHypothesis(model, 
                                hypothesis_matrix, 
                                hypothesis_values)
return(linear_hypothesis_result)}
                    \end{verbatim}
                    \item $H_0: \beta_1 = \beta_2, \beta_3 = \beta_4 \Leftrightarrow \beta_1 - \beta_2 = 0, \beta_3 - \beta_4 = 0$.
                    \begin{verbatim}
# Let us use the same technique as above

linear_hypothesis_test(model, "x1 = x2, x3 = x4")

linear_hypothesis_test <- function(model) {
                        
hypothesis_matrix <- matrix(c(0, 1, -1, 0, 0, 
                              0, 0, 0, 1, -1),
                nrow = 2, ncol = 5, byrow = TRUE)
                
hypothesis_values <- c(0, 0)

linear_hypothesis_result <- linearHypothesis(model, 
                                hypothesis_matrix, 
                                hypothesis_values)

return(linear_hypothesis_result)}
                    \end{verbatim}
                    \item $H_0: \beta_1 - 2\beta_2 = 4\beta_3$, $\beta_1 + 2 \beta_2 = 0 
                    \Leftrightarrow \beta_1 = -2\beta_2 + 4 \beta_3, \beta_1 = -2\beta_2$.
                    \begin{verbatim}
# Again, we will use the same technique as above
# but redefine the hypothesis matrix and values
linear_hypothesis_test(model, "x1 - 2*x2 = 4*x3,
                         x1 + 2*x2 = 0")

linear_hypothesis_test <- function(model) {

hypothesis_matrix <- matrix(c(1, -2, 0, 4, 0, 
                              1, 2, 0, 0, 0),
                nrow = 2, ncol = 5, 
                byrow = TRUE)

hypothesis_values <- c(0, 0)

linear_hypothesis_result <- linearHypothesis(model, 
                                hypothesis_matrix, 
                                hypothesis_values)

return(linear_hypothesis_result)}
                    \end{verbatim}
                \end{enumerate}

                \item Problem 3.1 on page 125
                \begin{enumerate}
                    \item 
                    \begin{verbatim}
library(MPV)
model_3.1 <- lm(y ~ x2 + x7 + x8, data = table.b1)
#####################
Call:
lm(formula = y ~ x2 + x7 + x8, data = table.b1)

Residuals:
    Min      1Q  Median      3Q     Max 
-3.0370 -0.7129 -0.2043  1.1101  3.7049 

Coefficients:
             Estimate Std. Error t value Pr(>|t|)    
(Intercept) -1.808372   7.900859  -0.229 0.820899    
x2           0.003598   0.000695   5.177 2.66e-05 ***
x7           0.193960   0.088233   2.198 0.037815 *  
x8          -0.004816   0.001277  -3.771 0.000938 ***
---
Signif. codes:  0 ‘***’ 0.001 ‘**’
0.01 ‘*’ 0.05 ‘.’ 0.1 ‘ ’ 1

Residual standard error: 1.706 on 24 degrees of freedom
Multiple R-squared:  0.7863,	Adjusted R-squared:  0.7596 
F-statistic: 29.44 on 3 and 24 DF,  p-value: 3.273e-08
                    \end{verbatim}
                    \item 
                    \begin{verbatim}
anova_3.1 <- anova(model_3.1)
#####################
Analysis of Variance Table

Response: y
          Df  Sum Sq Mean Sq F value    Pr(>F)    
x2         1  76.193  76.193  26.172 3.100e-05 ***
x7         1 139.501 139.501  47.918 3.698e-07 ***
x8         1  41.400  41.400  14.221 0.0009378 ***
Residuals 24  69.870   2.911                      
---
Signif. codes:  0 ‘***’ 0.001 ‘**’ 
0.01 ‘*’ 0.05 ‘.’ 0.1 ‘ ’ 1
                    \end{verbatim}
                    \item 
                    \begin{verbatim}
t_stats <- summary_model_3.1$coefficients[, "t value"]
#####################
(Intercept)          x2          x7          x8 
-0.228883    5.177090    2.198262   -3.771036 
                    \end{verbatim}
                    The conclusion we can draw about the roles the variables $x_2$, $x_7$, and $x_8$ play in predicting $y$ is that
                    $x_2$ and $x_7$ are significant predictors of $y$ because their $t$-statistics are greater than 2 in absolute value,
                    and their $p$-values are less than 0.05. $x_8$ is $\textbf{highly}$ significant predictor of $y$ because its $t$-statistic is less than -2.

                    $\therefore$ We reject all null hypotheses that $\beta_2 = 0$, $\beta_7 = 0$, and $\beta_8 = 0$.
                    \item  
                    \begin{verbatim}
r_squared <- summary_model_3.1$r.squared
#####################
[1] 0.7863069
adj_r_squared <- summary_model_3.1$adj.r.squared
#####################
[1] 0.7595953
                    \end{verbatim}
                    $\therefore$ The $R^2$ value is 0.7863 and the adjusted $R^2$ value is 0.7596.
                    \item \begin{verbatim}
model_reduced <- lm(y ~ x2 + x8, data = table.b1)
summary(model_reduced)
#####################
Call:
lm(formula = y ~ x2 + x8, data = table.b1)

Residuals:
    Min      1Q  Median      3Q     Max 
-2.4280 -1.3744 -0.0177  1.0010  4.1240 

Coefficients:
              Estimate Std. Error t value Pr(>|t|)    
(Intercept) 14.7126750  2.6175266   5.621 7.55e-06 ***
x2           0.0031111  0.0007074   4.398 0.000178 ***
x8          -0.0068083  0.0009658  -7.049 2.18e-07 ***
---
Signif. codes:  0 ‘***’ 0.001 ‘**’ 0.01 ‘*’ 0.05 ‘.’ 0.1 ‘ ’ 1

Residual standard error: 1.832 on 25 degrees of freedom
Multiple R-squared:  0.7433,	Adjusted R-squared:  0.7227 
F-statistic: 36.19 on 2 and 25 DF,  p-value: 4.152e-08
                    \end{verbatim}
                    \begin{verbatim}

model_full <- model_3.1

# Get the residual sum of squares (RSS) for both models
RSS_full <- sum(residuals(model_full)^2)
RSS_reduced <- sum(residuals(model_reduced)^2)

# Get the number of parameters 
# (coefficients) in the full and reduced models
p_full <- length(coef(model_full))  # including the intercept
p_reduced <- length(coef(model_reduced))  # including the intercept

# Get the degrees of freedom for the full model
df_full <- df.residual(model_full)

MSR_full <- (RSS_full - RSS_reduced) / (p_full - p_reduced)
MSR_reduced <- RSS_reduced / df_full

# Calculate the partial F-statistic
F_statistic <-  MSR_full / MSR_reduced
F_statistic
#####################
[1] 4.832354
                    \end{verbatim}
                    Using the adjusted $R^2$ value, we can see that the $R^2$ value for the full model is 0.7863, 
                    and the $R^2$ value for the reduced model is 0.7433. Which implies that the full model is better than the reduced model.

                    The partial F-statistic is 4.832354, and we know that $F = t^2$, then we have $t = 2.198262$. This matches 
                    the t-statistic for $x_7$ in the full model. This implies that the F-statistic and the t-statistic for $\beta_7$
                    are directly related.
                \end{enumerate}

        \item Problem 3.2 on page 125.
                Using the results of Problem 3.1, we can show numericaly that the square of the simple correlation coefficient
                between the observed values $y_i$ and the fitted values $\hat{y}_i$ is equal to the $R^2$.
                 Where $R^2$ is the coefficient of determination, and the formula is as follows
                \[R^2 = \frac{\text{SSR}}{\text{SST}} = 1 - \frac{\text{SSE}}{\text{SST}}\]
                where $\text{SSR}$ is the sum of squares of the regression, $\text{SST}$ is the total sum of squares, 
                and $\text{SSE}$ is the sum of squares of the error. The correlation coefficient $r$ 
                between $y_i$ and $\hat{y}_i$ is given by
                \[r = \frac{\text{Cov}(y, \hat{y})}{\sqrt{\text{Var}(y) \text{Var}(\hat{y})}}\]
                and the claim is that $r^2 = R^2$.

                \begin{verbatim}
model <- lm(y ~ x2 + x7 + x8, data = table.b1)

R_squared <- summary(model)$r.squared

y_fitted <- fitted(model)
y_observed <- table.b1$y
r <- cor(y_observed, y_fitted)
r_squared <- r^2
cat("R squared (from model): ", R_squared, "\n")
cat("R squared (from correlation): ", r_squared, "\n")
#####################
R squared (from model):  0.7863069 
R squared (from correlation):  0.7863069 
                \end{verbatim}
                $\therefore$ We have shown that the square of the simple correlation 
                coefficient $r^2$ between the observed values $y_i$ and the fitted values
                 $\hat{y}_i$ is equal to the $R^2$.
        \item Problem 3.3 on page 125.
                Referring to Problem 3.1, 
                \begin{enumerate}[a.]
                    \item Find a 95\% confidence interval for $\beta_2$. Let $\text{CI}$ be the confidence interval.
                    Then we need to find, 
                    \[ \text{CI} = \hat{\beta}_2 \pm t_{(0.5, n - p)} \times \text{SE}(\hat{\beta}_2)\]
                    where $\hat{\beta}_2$ is the estimate of $\beta_2$, $t_{\alpha/2, n - p}$ is the critical value of the $t$-distribution
                    \begin{verbatim}
coefficients <- summary(model)$coefficients
beta_2 <- coefficients["x2", "Estimate"]
SE_beta_2 <- coefficients["x2", "Std. Error"]

df <- df.residual(model)

t_crit <- qt(0.975, df)

CI_beta_2 <- c(beta_2 - t_crit * SE_beta_2, 
               beta_2 + t_crit * SE_beta_2)
#####################
[1] 0.002163664 0.005032477
                    \end{verbatim}
                    $\therefore$ The 95\% confidence interval for $\beta_2$ is $(0.002163664, 0.005032477)$.
                    \item Find a 95\% confidence interval on the mean number of games won
                    by a team when $x_2 = 2300$, $x_7 = 56.0$, and $x_8 = 2100$
                    \begin{verbatim}
# new obs
new_obs <- data.frame(x2 = 2300, x7 = 56.0, x8 = 2100)

# predict
y_hat <- predict(model, new_obs)

SE_y_hat <- predict(model, new_obs, se.fit = TRUE)$se.fit

CI_y_hat <- c(y_hat - t_crit * SE_y_hat, 
              y_hat + t_crit * SE_y_hat)
print(CI_y_hat)
#####################
1        1 
6.436203 7.996645 
                    \end{verbatim}
                    $\therefore$ The 95\% confidence interval for when $x_2 = 2300$, $x_7 = 56.0$, and $x_8 = 2100$ is $(6.436203, 7.996645)$.
                \end{enumerate}
        \item Problem 3.4 on page 126. Remodling table.b1 using, $x_7$ and $x_8$ as the predictors, and $y$ as the response.
            \begin{enumerate}[a.]
                \item Test for significane of regression 
                \begin{verbatim}
new_model <- lm(y ~ x7 + x8, data = table.b1)
#####################
Call:
lm(formula = y ~ x7 + x8, data = table.b1)

Residuals:
    Min      1Q  Median      3Q     Max 
-3.7985 -1.5166 -0.5792  1.9927  4.5248 

Coefficients:
             Estimate Std. Error t value Pr(>|t|)   
(Intercept) 17.944319   9.862484   1.819  0.08084 . 
x7           0.048371   0.119219   0.406  0.68839   
x8          -0.006537   0.001758  -3.719  0.00102 **
---
Signif. codes:  0 ‘***’ 0.001 
‘**’ 0.01 ‘*’ 0.05 ‘.’ 0.1 ‘ ’ 1

Residual standard error: 2.432 on 25 degrees of freedom
Multiple R-squared:  0.5477,	Adjusted R-squared:  0.5115 
F-statistic: 15.13 on 2 and 25 DF,  
p-value: 4.935e-05
                \end{verbatim}
                $\therefore$ The regression is \textbf{highly} significant because the p-value $4.935 \times 10^{-5} < 0.05$
                \item Calculate the $R^2$ and adjusted $R^2$ values
                \begin{verbatim}
R_squared <- summary(new_model)$r.squared
adj_R_squared <- summary(new_model)$adj.r.squared
cat("R squared: ", R_squared, "\n")
cat("Adjusted R squared: ", adj_R_squared, "\n")
#####################
R squared:  0.547682
Adjusted R squared:  0.5114655
                \end{verbatim}
                $\therefore$ These $R^2$ and adjusted $R^2$ values computed compared to 3.1 are lower, signifying
                that the model did not fit as well as the previous model.
                \item Calcuate 95\% CI on $\beta_7$ and find 95\% CI on the mean number of games won by 
                a team when $x_7 = 56.0$ and $x_8 = 2100$.
                \begin{verbatim}
coefficients <- summary(new_model)$coefficients
beta_7 <- coefficients["x7", "Estimate"]
SE_beta_7 <- coefficients["x7", "Std. Error"]

df <- df.residual(new_model)

t_crit <- qt(0.975, df)

CI_beta_7 <- c(beta_7 - t_crit * SE_beta_7, 
               beta_7 + t_crit * SE_beta_7)
print(CI_beta_7)
#####################
[1] -0.1971643  0.2939060

# new obs
new_obs <- data.frame(x7 = 56.0, x8 = 2100)

# predict
y_hat <- predict(new_model, new_obs)

SE_y_hat <- predict(new_model, 
            new_obs, se.fit = TRUE)$se.fit

CI_y_hat <- c(y_hat - t_crit * SE_y_hat,
            y_hat + t_crit * SE_y_hat)
print(CI_y_hat)
#####################
1        1 
5.828643 8.023842 
                \end{verbatim}
                $\therefore$ The 95\% confidence interval for $\beta_7$ is $(-0.1971643, 0.2939060)$, 
                and the 95\% confidence interval for when $x_7 = 56.0$ and $x_8 = 2100$ is $(5.828643, 8.023842)$.
                \item Conclusions we can draw from this problem of omitting an important regressor $x_2$ from the model, is that
                the model performed significantly worst, as the $R^2$ and adjusted $R^2$ values are lower than the previous model 
                and the confidence intervals are wider, which means that the estimates are less precise.
            \end{enumerate}
            \item Bonus Problem
            \begin{enumerate}
                \item Show that $SSR(\btilde_2 | \btilde_1) =  \etilde^\T (\HM - \HM_1) \etilde$, if $H_0: \beta_2 = 0$ is true.
                \item Show that $\frac{SSR(\btilde_2 | \btilde_1)}{\sigma^2} \sim \chi^2_{(r)}$.
                \item Show that $\frac{SSR(\btilde_2 | \btilde_1)}{MSE} \sim F(r, n - p)}$.
            \end{enumerate}
                
                
                 
    \end{enumerate}
End of Assignment 3.
\end{document}