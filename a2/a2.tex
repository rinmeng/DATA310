\documentclass[12pt]{article}
\usepackage{enumerate}
\usepackage{amsmath}
\usepackage{amsthm}
\usepackage{amssymb}
\usepackage{changepage}

\title{Assignment 2}
\author{Rin Meng \\ Student ID: 51940633}

\begin{document}
\maketitle

\begin{enumerate}[1.]
    \item Given the summary output, it is true that:
    \begin{enumerate}[(a)]
        \item \[\text{F-Statistic} = 52.77\]
        \[\text{MSE} = (\text{Residual Standard Error})^2 = 2.792^2 =  7.80\]
        \item To build the anova table, we need to do calculations as follows:
        
            Finding the degrees of freedom:
            \[ DF_{Reg} = 1 \]
            \[ DF_{Error} = 7 \]
            \[ DF_{Total} = 8 \]
            \[ SS_{Total} = SS_{Reg} + SS_{Error}\]
            Calculate the Mean Squares Reg and Error:
            \[ F = \frac{MS_{Reg}}{MS_{Error}} = 52.77\]
            \[ MS_{Error} = MSE = 7.80\]
            \[ MS_{Reg} = F \cdot MS_{Error} = 52.77 \cdot 7.80 = 411.34\]
            Calculate the Sum Squares Reg and Error:
            \[ MS_{Reg} = \frac{SS_{Reg}}{DF_{Reg}} \Leftrightarrow SS_{Reg} = MS_{Reg} \cdot DF_{Reg}\]
            \[ MS_{Error} = \frac{SS_{Error}}{DF_{Error}} \Leftrightarrow SS_{Error} = MS_{Error} \cdot DF_{Error}\]
            \[ SS_{Total} = SS_{Reg} + SS_{Error}\]
            \[ SS_{Reg} = MS_{Reg} \cdot DF_{Reg} = 411.34 \cdot 1 = 411.34\]
            \[ SS_{Error} = MS_{Error} \cdot DF_{Error}= 7.80 \cdot 7 = 54.60 \]
            \[ SS_{Total} = SS_{Reg} + SS_{Error} = 411.34 + 54.60 = 465.94\]
            \[ MS_{Total} = \frac{SS_{Total}}{DF_{Total}} = \frac{465.94}{8} = 58.24 \]
            \begin{center}
                \begin{tabular}{|c|c|c|c|c|}
                    \hline
                    Source & DF & SS & MS & F \\
                    \hline
                    Reg. & 1 & 411.34 & 411.34 & 52.77 \\
                    Error & 7 & 54.60 & 7.80 &  \\
                    Total & 8 & 465.94 & 58.24 & \\
                    \hline
                \end{tabular}
            \end{center}
        \item The anova function returns:
            \begin{verbatim}
Analysis of Variance Table

Response: R
        Df Sum Sq Mean Sq F value    Pr(>F)    
W          1 411.42  411.42  52.767 0.0001679 ***
Residuals  7  54.58    7.80                      
---
Signif. codes:  0 ‘***’ 0.001 ‘**’ 0.01 ‘*’ 0.05 ‘.’ 0.1 ‘ ’ 1
            \end{verbatim}
            Our hand calculations are fairly consistent with the output from the anova
            function in R, minus a few rounding errors.
        \item Calculating the $\sqrt{F}$ value:
            \[ \sqrt{F} = \sqrt{52.77} = 7.27 \]
            The t value for $\hat{\beta}_1$ is 7.264, which is very close to the $\sqrt{F}$ 
            value. This is expected because, for simple linear regression with one 
            predictor, the square of the t-value for the slope is equal to the 
            F-statistic 
            \[ \sqrt{F} = t \Leftrightarrow F = t^2 \]
    \end{enumerate}
    
    \item Given
        \begin{enumerate}
            \item 
                \begin{enumerate}[-]
                    \item \textbf{First task time}: $\epsilon$ is 
                    exponentially distributed with mean $ \frac{1}{\lambda} $, so $E[\epsilon] = \frac{1}{\lambda}$.
                    \item \textbf{Second task time}: Proportional to $x$, with a proportionality constant $\beta$. So the time required 
                    should be $\beta x$.
                    \item \textbf{Total time}: sum of times it takes to complete the two tasks
                    impying that the total time is $y = \beta x + \epsilon$.
                \end{enumerate}
            
                Then the final linear model would be 
                \[y = \beta x + \epsilon \]

                \item To derive the maximum likelihood estimator for $\beta$ and $\lambda$, we need to find the pdf of the exponential distribution.
                \[f(\epsilon) = \lambda e^{-\lambda \epsilon} \text{ for } x \geq 0\]
                \[\Rightarrow f(y) = \lambda e^{-\lambda (y - \beta x)} \text{ for } y \geq \beta x\]

                Deriving maximum likelihood estimator for $\beta$:
                \begin{align*}
                    L(\beta, \lambda) &= \prod_{i=1}^{n} \lambda e^{-\lambda (y_i - \beta x_i)} \\
                    \log (\beta, \lambda) &= \ell (\beta, \lambda) = \sum_{i=1}^{n} \log (\lambda e^{-\lambda (y_i - \beta x_i)}) \\
                    \frac{\partial \ell (\beta, \lambda)}{\partial \beta} &= \sum_{i=1}^{n} \frac{\partial}{\partial \beta} \left( \log (\lambda e^{-\lambda (y_i - \beta x_i)}) \right) \\
                    \frac{\partial \ell (\beta, \lambda)}{\partial \beta} &= \sum_{i=1}^{n} \frac{\partial}{\partial \beta} \left( \log \lambda - \lambda (y_i - \beta x_i)  \right) \\
                    \frac{\partial \ell (\beta, \lambda)}{\partial \beta} &= \sum_{i=1}^{n} -x_i \lambda + x_i \lambda \beta \\
                    \frac{\partial \ell (\beta, \lambda)}{\partial \beta} &= \sum_{i=1}^{n} -x_i \lambda + x_i \lambda \beta = 0 \\
                    \sum_{i=1}^{n} x_i \lambda &= \sum_{i=1}^{n} x_i \lambda \beta \\
                    \beta &= 1
                \end{align*}
                Deriving maximum likelihood estimator for $\lambda$:
                \begin{align*}
                    L(\beta, \lambda) &= \prod_{i=1}^{n} \lambda e^{-\lambda (y_i - \beta x_i)} \\
                    \log L(\beta, \lambda) &= \ell (\beta, \lambda) = \sum_{i=1}^{n} \log \lambda e^{-\lambda (y_i - \beta x_i)} \\
                    \frac{\partial \ell (\beta, \lambda)}{\partial \lambda} &= \sum_{i=1}^{n} \frac{\partial}{\partial \lambda} \left(  \log \lambda e^{-\lambda (y_i - \beta x_i)} \right) \\
                    \frac{\partial \ell (\beta, \lambda)}{\partial \lambda} &= \sum_{i=1}^{n} \frac{\partial}{\partial \lambda} \left( \log \lambda - \lambda (y_i - \beta x_i) \right) \\
                    \frac{\partial \ell (\beta, \lambda)}{\partial \lambda} &= \sum_{i=1}^{n} \frac{1}{\lambda} - (y_i - \beta x_i) \\
                    \frac{\partial \ell (\beta, \lambda)}{\partial \lambda} &= \sum_{i=1}^{n} \frac{1}{\lambda} - (y_i - \beta x_i) = 0 \\
                    \sum_{i=1}^{n} \frac{1}{\lambda} &= \sum_{i=1}^{n} (y_i - \beta x_i) \\
                    \frac{n}{\lambda} &= \sum_{i=1}^{n} (y_i - \beta x_i) \\
                    \lambda &= \frac{n}{\sum_{i=1}^{n} (y_i - \beta x_i)}
                \end{align*}

                $\therefore$ The maximum likelihood estimator for $\beta$ is 1 and $\lambda$ is $\frac{n}{\sum_{i=1}^{n} (y_i - \beta x_i)}$.
                
        \end{enumerate}
    
\end{enumerate}
End of Assignment 2.
\end{document}
